\chapter{Definition}
\section{Preisbildung}
\label{sec:preisbildung}
Der Begriff Preisbildung lässt sich als Prozess definieren, bei dem sich ein Unternehmen auf einen Preis für eine Dienstleistung oder ein Produkt festlegt. Das Ziel des Prozesses ist einen Preis zu finden, der sowohl für das Unternehmen als auch für seine Kunden akzeptabel ist und den Gewinn des Unternehmens steigert.

Die zwei wichtigsten Faktoren für die Bildung des Preises sind die Deckung der Kosten und die Marktfähigkeit des Preises. \autocite[142]{Schinnerl2021}

Dabei beziehen sich die zu deckenden Kosten auf die direkten Kosten, die bei der Produktion des Gutes entstanden sind, (variable Kosten) und die indirekten Kosten, die unabhängig von der Produktion anfallen (fixen Kosten).

Zu den fixen Kosten zählt beispielsweise die Miete von Bürogebäuden, während zu den variablen Kosten der Einkaufspreis der Rohstoffe und Personalnebenkosten, zählt

Die Marktfähigkeit des Preises ermittelt sich mithilfe von Konkurrenzanalysen, Zielgruppenanalyse aber auch durch Betrachtung des Marktpreises und der Produkteigenschaften, insbesondere im Vergleich mit der Konkurrenz.

So lässt sich beispielsweise ein Preis, welcher höher als der Marktpreis liegt, durch eine bessere Produkt-Qualität oder eine nachhaltigere Fertigung rechtfertigen. Mithilfe der Zielgruppenanalyse, kann in Erfahrung gebracht werden, was die potenziellen Kunden wollen und so besser auf ihre Bedürfnisse und Anforderungen preislich reagiert werden.

Die Kombination aus den fixen und den variablen Kosten nennt man Selbstkosten und diese stellen im Zusammenhang mit dem Produkt die Preisuntergrenze dar. Diese Preisuntergrenze darf nicht unterschritten werden, da sonst das Unternehmen mit dem Verkauf des Produktes keinen Gewinn erwirtschaften kann. Die Differenz aus dem tatsächlichen Verkaufspreis und der Preisuntergrenze (Selbstkosten) ist dann der tatsächliche Gewinn den das Unternehmen mit dem Verkauf eines einzelnen Produktes verbuchen kann (Gewinnaufschlag). 


\section{Marktpreis}
\label{sec:marktpreis}
Der Marktpreis ist der durchschnittliche Preis zu dem ein bestimmtes Gut zu einem bestimmten Zeitpunkt am Markt gehandelt wird. Weicht man mit dem Preis für sein Gut zu sehr vom Marktpreis ab, kann dies negative Folgen auf den Gewinn haben den man mit dem Gut erzielt. \autocite[142]{Schinnerl2021}
 

In einem vollkommenen Markt stimmt der Marktpreis mit dem Gleichgewichtspreis überein. \autocite[132]{Forner2022}

Dieser Gleichgewichtspreis ist der Preis, bei dem Angebot und Nachfrage sich genau decken. \autocite[178]{Forner2022}

Allerdings ist der vollkommene Markt in der Praxis nur sehr selten zu finden. Dieser wird hauptsächlich in der Theorie genutzt um Zusammenhänge zwischen Angebot und Nachfrage bei der Preisbildung in verschiedenen Märkten zu untersuchen.\autocite[93]{Pollert2013}

Mit dem vollkommenen Markt als Grundlage kann untersucht werden, wie sich der Marktpreis je nach Anzahl der verschiedenen Anbieter und Nachfrager verändert. 

Es gibt auf jeder Marktseite (Angebot und Nachfrage) drei Konstellationen die auftreten können: Monopol, Oligopol und Polypol. Das Polypol beschreibt die Existenz von vielen Marktteilnehmern, das Oligopol beschreibt die Existenz von einigen wenigen Marktteilnehmern und beim Monopol gibt es nur einen einzigen Marktteilnehmer auf einer Marktseite.

So ergeben sich insgesamt neun Konstellationen, aufgezeigt in \autoref{fig:marktdefinitionen}.

Eine geringere Anzahl von Teilnehmern bedeutet mehr Markteinfluss für die Teilnehmer auf der entsprechenden Marktseite. Je weniger Anbieter desto grö\ss{}er ist der Einfluss der einzelnen Anbieter auf den Marktpreis, was sich wiederum auf den Markt auswirkt. Der Einfluss auf den Markt von einzelnen Nachfragern erhöht sich ebenso in Abhängigkeit von der Anzahl.

Nehmen wir beispielsweise an, dass es nur einen Anbieter für Strom in einer Region gibt. Dann müssen sich die Kunden nach dem Preis richten, den der Stromanbieter festsetzt, oder auf Strom verzichten. Das selbe Beispiel gilt für Nachfrager. Wenn es für ein bestimmtes Produkt nur einen Abnehmer gibt, kann dieser Druck auf die Anbieter ausüben, um das Produkt günstiger erwerben zu können.

\begin{figure}
  \begin{center}
    \begin{tabular}{|c|c|c|c|}
      \hline
      \thead{\makecell{Nachfrager\\\hline{}Anbieter}} & \thead{Viele} & \thead{Wenige} & \thead{Einer}  \\
      \hline
      \hline
      \thead{Viele} & Polypol & Angebotsoligopol & Angebotsmonopol \\
      \hline
      \thead{Wenige} & \makecell{Nachfrageoligopol \\ (Oligopson)} & \makecell{Zweiseitiges \\ Oligopol} & \makecell{Beschränktes \\ Angebotsmonopol} \\
      \hline
      \thead{Einer} & \makecell{Nachfragemonopol \\ (Monopson)} & \makecell{Beschränktes \\ Nachfragemonopol} & \makecell{Zweiseitiges \\ Monopol} \\
      \hline
    \end{tabular}
  \end{center}
  \caption{Übersicht über die Marktformen}
  \label{fig:marktdefinitionen}
  \medskip
  \small
  Übersicht über die Marktformen, entnommen und leicht bearbeitet von \autocite[132]{Forner2022}
\end{figure}


% Wichtig zu erwähnen ist dass für die 
%
%   Ist das Gut einzigartig und wird nur vom preisbildenden Unternehmen angeboten, spricht man von unvollständigem Wettbewerb und einem Angebotsmonopol. Das Unternehmen hat in diesem Fall mehr Freiheit was die Festlegung auf einen Verkaufspreis betrifft, da es keine Konkurrenten gibt die Einfluss auf den Marktpreis nehmen.
%
%   Sind hingegen viele Anbieter für ein Produkt am Markt, spricht man von vollständigem Wettbewerb und einem Angebotspolypol. In diesem Fall ist die Freiheit des Unternehmens bei der Preisbildung eingeschränkt.

\chapter{Konzept/Strategie}

Es gibt verschiedene Strategien um einen Preis für ein Produkt festzulegen. Unabhängig von der Strategie gibt es jedoch üblicherweise einen gewissen Preis-Spielraum. Wie in \nameref{sec:preisbildung} bereits erläutert kann man die Selbstkosten des Gutes als die Preisuntergrenze sehen

Einige der gängigsten Strategien zur Preisbildung sind:

\begin{itemize}
  \item kostenorientierte Preisbildung
  \item Nachfrageorientierte Preisbildung
  \item Wettbewerbsorientierte Preisbildung
  \item Preisdifferenzierung
\end{itemize}

\section{Kostenorientierte Preisbildung}

Bei der Kostenorientierten Preisbildung ist der Verkaufspreis des Produktes abhängig von den Produktionskosten. Auf die variablen Stückkosten wird eine Gewinnmarge addiert um den Verkaufspreis zu berechnen.

Diese Art der Preisbildung ist sehr einfach, da man für die Berechnung des Produktpreises nur die Selbstkosten des Produktes und den Prozentsatz der Gewinnmarge benötigt.

\section{Nachfrageorientierte Preisbildung}
Bei der nachfrageorientierten Preisbildung wird die Nachfrage der Kunden nach dem Produkt und deren Reaktion auf Änderungen zur Festlegung auf einen Preis berücksichtigt.

Nimmt ein Unternehmen für ein Produkt beispielsweise 200\euro{}, kann es sein dass es mit einer Preiserhöhung auf 220\euro{} keinen merklichen Rückgang der Nachfrage gibt. Es wird also versucht den Preis zu finden bei dem der Gewinn maximal ist.

\section{Wettbewerbsorientierte Preisbildung}

Bei der wettbewerbsorientierten Preisbildung orientiert sich das Unternehmen an der Konkurrenz. So wird der Marktpreis als Leitfaden betrachtet an dem sich die gesamte Preisstrategie ausrichtet. Wenn das Unternehmen sich beispielsweise zum Ziel genommen hat die Konkurrenz immer zu unterbieten, muss das Unternehmen den Preis immer geringer als den Marktpreis halten. Wichtig ist es aber die Preisuntergrenze bei dem Verfahren nicht zu unterschreiten, da sonst das Unternehmen Verlust macht mit dem Vertrieb des Produktes.

\section{Preisdifferenzierung}

Die Strategie der Preisdifferenzierung ist es ein Produkt zu verschiedenen Preisen anzubieten. Der Preis wird hier anhand verschiedener Faktoren bestimmt:
\begin{itemize}
  \item Saisonal (Bsp. Sommerschlussverkauf)
  \item Nachfrager-abhängig (Bsp. Frauenhaarschnitt teuerer als Männerhaarschnitt)
  \item Qualitativ (Bsp. iPhone, iPhone Pro, iPhone Pro Max)
  \item Quantitativ (Bsp. Mengenrabatt)
\end{itemize}

Das Ziel der Strategie ist sich an die Bedürfnisse und Preferenzen der Nachfrager anzupassen. So kann man zum Beispiel Flüge während der Ferienzeit etwas teurer ansetzen, da beispielsweise Familien mit Kindern, die zeitlich an die Ferien gebunden sind, oft nur diesen Zeitraum für die Urlaubsplanung wählen können.

% Setzen wir die definierten Konzepte nun beispielhaft für ein fiktives Unternehmen um.
%
% Die ABC GmbH möchte in der Region Musterhausen Brote aus eigener Herstellung in einem kleinen Laden anbieten.
%
% Eine Umfrage hat ergeben, dass die Bürger in der Region ihre Brote bisher alle in dem Supermarkt in Musterhausen für einen Marktpreis von circa 3,50\euro{} eingekauft haben. Die Umfrage hat ebenfalls ergeben, dass circa die Hälfte der Leute auf der Suche nach einer besseren Alternative ist, es aber bisher keine gibt da der Supermarkt der einzige Anbieter in der Region ist. Auch haben die meisten Bürger eine genaue Vorstellung davon wie viel ein Brot maximal kosten darf, der maximal verträgliche Preis liegt bei 5\euro{}.
%
% Die ABC GmbH hat bereits eine Aufstellung über die fixen und variablen Kosten erstellt und möchte nun einen geeigneten Preis für ihr Brot finden. Die Selbstkosten für die Herstellung des Brotes liegen bei circa 2,00\euro{} pro Stück wenn pro Monat 2000 Brote verkauft werden. Um 2000 verkaufte Brote zu erreichen müsste circa ein Viertel der Leute in der Region zu Stammkunden werden und ihre Brote nur noch bei der Bäckerei der ABC GmbH kaufen.
%
% Die ABC GmbH möchste also nun einen angemessenen Preis für das Brot verlangen, Sie hat dafür den Bereich von 2\euro{} (Preisuntergrenze) und 5\euro{} (maximal verträglicher Preis - Bürger) festgelegt.
%
% Da die ABC GmbH sich nach Markteintritt in einem Angebotsoligopol befindet, liegt es nahe sich am bestehenden Marktpreis zu orientieren. Denn der Supermarkt hat zu Beginn die deutlich höhere Marktmacht 
\chapter{Chancen/Risiken}
\section{Chancen}
Mit einer effektiven Preisstrategie kann ein Unternehmen:
\begin{itemize}
  \item Den Gewinn steigern
  \item Wettbewerbsfähig bleiben
  \item Den eigenen Marktanteil erhöhen
\end{itemize}

\subsection{Den Gewinn steigern}
Den richtigen Preis für ein Produkt zu bestimmen, zahlt sich für Unternehmen aus. 

Zum Beispiel wird ein Produkt am Markt für 600\euro{} gehandelt und verkauft sich 1000-mal. 
Wenn das exakte gleiche Produkt am Markt für 700\euro{} gehandelt werden würde und sich immer noch 900-mal verkaufen würde, wäre der Umsatz bei der zweiten Variante um 30000\euro{} höher. 

\subsection{Wettbewerbsfähig bleiben}
Um am Markt nicht von Konkurrenten abgehängt zu werden, sollte der Preis des Produktes immer auf die am Markt gegebenen Bedingungen und die Bedürfnisse der Kunden angepasst werden. 

\subsection{Den eigenen Marktanteil erhöhen}
Ein guter Ruf bei den Kunden lässt sich auch mithilfe von guter Preisbildung erreichen. Das Vertrauen der Kunden spiegelt sich auf dem Konto des Unternehmens wieder.

\clearpage{}

\section{Risiken}

Allerdings kann eine schlechte Preisbildung auch Nachteile mit sich bringen:
\begin{itemize}
  \item Unvorhergesehene Folgen
  \item Verlust von Kundenvertrauen
  \item Fehlende Deckung der Kosten
  \item Veraltete Datengrundlage
\end{itemize}

\subsection{Unvorhergesehene Folgen}
Die Preisbildung kann sehr komplex werden. Die Faktoren die theoretisch berücksichtigt werden können sind zahlreich. So kann es sein, dass mit den Informationen die vorliegen eine Fehlinformation trifft.

\subsection{Verlust von Kundenvertrauen}
Rapide Preisänderungen könnten Kunden abschrecken und verwirren. Zudem könnten Kunden bei zu starkem Preisanstieg denken, dass das Unternehmen gierig ist und dass der Kunde ausgenutzt wird. 
Auch wenn an der Qualität gespart wird bei gleichbleibendem Preis, könnten Kunden sich betrogen fühlen und zur Konkurrenz wechseln. 

\subsection{Fehlende Deckung der Kosten}
Die untere Preisgrenze ist durch die Selbstkosten des Produktes festgelegt. Orientiert man sich bei der Preisbildung falsch und die Konkurrenz legt Preise fest, die die eigene Preisgrenze unterschreiten, kann es zu Problemen kommen. Wenn der Preis nicht so niedrig wie der der Konkurrenz ist, läuft man Gefahr dass die Kundschaft zu der Konkurrenz übergeht. Wenn man den Preis unter die Preisgrenze setzt, macht man Verlust mit dem Verkauf des Produktes.
\subsection{Veraltete Datengrundlage}
Die Daten die für die Preisbildung relevant sind, sollten regelmä\ss{}ig aktualisiert werden. Das selbe gilt für die Preisbildung. 

Der Markt kann sich innerhalb kürzester Zeitabstände verändern. Abhängig von den Änderungen kann es auch nötig sein, den Preis seiner Produkte anzupassen. 
